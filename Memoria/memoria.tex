\documentclass[a4paper]{article}

\usepackage[utf8]{inputenc}
\usepackage[spanish]{babel}
\usepackage[pdftex]{color,graphicx}
\usepackage{caption}
\usepackage{amsmath}

\usepackage[pdftex,
	colorlinks=true,
	pdfauthor={Alejandro Vicario, Juan José Alberca},
	pdftitle={}]{hyperref}

\graphicspath{ {curvas/} }

\begin{document}

\title{Entregable 3 \\
\large TeleLabo 2018. Diseño de controladores}
\author{
	Juan José Alberca\\
	\texttt{jj.alberca@alumnos.upm.es}
	\and
	Alejandro Vicario\\
	\texttt{a.vicarioe@alumnos.upm.es}
}
\date{\today}


\maketitle


\section{Diferencia entre controladores}
\subsection{Seguimiento de señales de referencia monómicas}
En la función de transferencia de los controladores de segundo orden (P, P-D y PD), se puede comprobar que el numerador contiene un subpolinomio completo del denominador con grado $n_s=0$. Esto quiere decir que solo se resuelve el problema de seguimiento para señales monómicas de grado $q<1$, es decir, la señal escalón. Esto quiere decir que:
\begin{equation}
	e(\infty)=\lim_{s \rightarrow 0}
	s H(s) U(s) = 0
\end{equation}

Siendo $e(s)$ el error en la salida cuando en la entrada tenemos la función escalón $U(s)$. Con funciones de grado superior no se cumple esto: para la señal rampa se obtiene un error finito y para señales de mayor grado, error infinito.

Los controladores de tercer grado con un grado de libertad, que son PI, PID, PI-D, tienen un subpolinomio completo del denominador de grado $n_s=1$, por lo que además de poder resolver el problema de seguimiento con la señal escalón, también lo podrán resolver con la señal rampa. Es decir, lo podrán resolver para señales monómicas de grado $q<2$.
\begin{equation}
e(\infty)=\lim_{s \rightarrow 0}
s H(s) R(s) = 0, \textrm{con grado de } R(s)<2. 
\end{equation}

Para los sistemas de dos grados de libertad estudiados existen dos posibilidades. En el sistema PID-D, al ser una generalización de los sistemas de tercer orden ya estudiados posee las mismas propiedades en el sentido del seguimiento de señales de referencia monómicas, es decir, hay error nulo para la rampa y error finito para la parábola. Pero si este sistema cumple la siguiente condición:
\begin{equation}
	K_P \tau_{D2} = -\frac{p}{K}
\end{equation}
también resuelve el problema de seguimiento para señales parabólicas.

 Por otro lado, el sistema D\textbar PID si es capaz de resolver parábola ya que el numerador contiene un subpolinomio completo del denominador con grado $n_s=3$.

\subsection{Comportamiento en régimen transitorio}
Para estudiar de forma más sencilla los controladores, en lugar de utilizar el parámetro de la constante proporcional, $K_p$, la constante de tiempo integral, $\tau_I$ y la constante de tiempo derivativa, $\tau_D$, se utilizan otros parámetros derivados de estos que resultan más intuitivos.
\begin{itemize}
	\item $\omega_n$: pulsación natural.
	\item $\zeta$: coeficiente de amortiguamiento. Se relaciona la velocidad con la que el sistema responde a una perturbación y con la amplitud de las oscilaciones que se pueden producir.
	\item $\beta$ y $\beta_2$: se relaciona con el tiempo que el sistema tarda en estabilizarse.
	\item $M_p$: sobreelongación máxima.
	\item $t_r$: tiempo de subida.
	\item $t_p$: tiempo de pico.
	\item $t_s$: tiempo de estabilización.
\end{itemize}
\subsubsection{Controladores P y P-D}
En un controlador P-D tenemos el parámetro de la rama proporcional $K_p$ y la constante de tiempo derivativa $\tau_D$. Su relación con $\beta_2$ y con $\zeta$ son las siguientes:
\begin{subequations}
	\begin{equation}
		\zeta= \frac{p+K_p K \tau_D}{\sqrt{2 K_p K}}
	\end{equation}
	\begin{equation}
		\tau_D= \frac{\zeta (2-\beta_2)}{\sqrt{K_p K}}
	\end{equation}
	\label{eq:1}
\end{subequations}
Como se puede observar, cuando $\beta_2=0$, se anula la rama derivativa, por lo que se comprueba que el controlador P es un caso particular del controlador P-D. Por lo tanto, comparten las características de régimen transitorio. Estas características corresponden con las de un sistema de segundo orden en su primera forma canónica. Debido a esto, en el controlador P-D es posible controlar $M_p$ de forma independiente a $t_p$, $t_r$ y $t_s$ mediante la variación de $K_p$ y $\tau_D$ según la ecuación \ref{eq:1}.

\subsubsection{Controladores PI, PID y PI-D}
Los controladores P, PD, P-D y PI son casos particulares de los controladores PID y PI-D. La razón de que el P y el PD se estudien por separado se debe a que su estudio matemático es posible y no es obligatorio realizar simulaciones, como ocurre con los otros.
Con estos controladores aparece el parámetro $\beta$ que procede de una escritura alternativa del polinomio característico de las funciones de transferencia de los mismos.

Las ganancias de los factores Derivativo e Integral de estos controladores vienen dadas por:
\begin{subequations}
	\begin{equation}
	K_D=K_P \tau_D
	\end{equation}
	\begin{equation}
	K_I=\frac{K_P}{\tau_I}
	\end{equation}
\end{subequations}

Ya que estos controladores no se pueden analizar de forma matemática es importante entender como afectan los parámetros del sistema $K_P$, $\tau_I$ y $\tau_D$ a sus polos. Estos nos darán la información sobre su comportamiento. Cuanto más lejos del origen se encuentre el polo, más rápido será el sistema respecto a cualquier cambio y cuanto más alejados entre sí estén los polos conjugados el controlador será más inestable y mayor será $M_p$. Este comportamiento está relacionado con la magnitud del coeficiente de amortiguamiento $\zeta$.

\subsubsection{Controladores PID-D y D\textbar PID}
Como los controladores PID-D son una generalización de los controladores PID y PI-D, para los que con $\tau_{D1}=0$ se obtiene un sistema PI-D y con $\tau_{D2}=0$ se obtiene un sistema PID.

Para analizar estos controladores se introducen los parámetros $r1$, $r2$, $r3$ que son funciones que dependen de $\beta$, $\zeta$ y $\omega_n$ de tal forma que si $frac{r1}{\omega_n}$, $frac{r2}{\omega_n^2}$, $frac{r3}{\omega_n}$ son independientes de $\beta_2$, como en el caso del controlador PID-D y D\textbar PID, tendremos $M_p$ independiente de $\beta$, por lo que para diseñar estos controladores se puede seguir el mismo procedimientos que para diseñar los controladores PID y los PI-D.

\section{Análisis, diseño e implementación de un sistema de control de la posición angular de un motor DC, imponiendo especificaciones de régimen permanente y transitorio, utilizando un controlador D\textbar PID}
Se va a utilizar la siguiente función de transferencia:
\begin{equation}
G(s)=frac{2652.28}{s(s+64.986)}
\end{equation}
\subsection{Análisis de estabilidad}
Para estudiar la estabilidad del sistema primero debemos obtener el polinomio característico. Para un D\textbar PID es:
\begin{equation}
P(s)=s^3+(p+K K_P\tau_{D1})s^2+K K_P(s+\frac{1}{\tau_I})
\end{equation}
Donde:
\begin{itemize}
	\item $p$ es el polo del motor
	\item $K=K'/R$ con $R=23.04$ de la reductora
	\item $K_P$ es la constante de la rama proporcional
	\item $\tau_{D1}$ es la constante de tiempo derivativa de la rama principal
	\item $\tau_I$ es la constante de tiempo integral
\end{itemize}

Con el polinomio característico construimos la tabla de Routh:
\begin{figure}[!h]
	\begin{center}
		\large
		\begin{tabular}{|c c c|}
			\hline 
			$s^3:$ & 1 & $K K_P$ \\ 
			\hline 
			$s^2:$ & $p+K K_P \tau_{D1}$ & $\frac{K K_P}{\tau_I}$ \\ 
			\hline 
			$s:$ & $\frac{K K_P (p+K K_P \tau_{D1} -\frac{1}{\tau_I})}{p+K K_P \tau_{D1}}$ &  \\ 
			\hline 
			$s^0:$ & $\frac{K K_P}{\tau_I}$ &  \\ 
			\hline 
		\end{tabular}
	\caption{Tabla de Routh de D\textbar PID}
	\label{tabla1}
	\end{center}
\end{figure}

La condición necesaria para que el sistema sea estable es que no debe haber cambios de signo en la primera columna de la Tabla de Routh (Figura \ref{tabla1}). Analizando los términos de la tabla se llegan a las siguientes condiciones:

Como $1$ es positivo, por lo que el resto de parámetros deben ser positivos.
\begin{equation}
	p+K K_P \tau_{D1}>0
\end{equation}
Reorganizando, 
\begin{equation}
	-\frac{p}{K}<K_P \tau_{D1}
\end{equation}
También debe cumplirse
\begin{equation}
	\frac{K K_P (p+K K_P \tau_{D1} -\frac{1}{\tau_I})}{p+K K_P \tau_{D1}} > 0
\end{equation}
O lo que es lo mismo,
\begin{equation}
	\frac{1}{\tau_I} > p+K K_P \tau_{D1}
\end{equation}
Además se debe cumplir $\tau_I > 0$.

\subsection{Problema de seguimiento de las señales escalón, rampa y parábola}
La función de transferencia del controlador D\textbar PID tiene en su numerador un subpolinomio completo del denominador de grado 1. Por lo tanto, resuelve el problema de seguimiento para las funciones de grado $q<2$, es decir, para el escalón y la rampa.

Para comprobar que el controlador resuelve el problema de seguimiento para una señal determinada $R(s)$, se aplica el teorema del valor final sobre la señal de error del sistema.
La función de transferencia del error es,
\begin{equation}
	H_{e, D|PID}(s)=\frac{s^2(s+p-K K_P \tau_{D2})}{s^2(s+p)+K K_P \tau_{D1}(s^2+\frac{s}{\tau_{D1}}+\frac{1}{\tau_{D1} \tau_I})}
\end{equation}

Y tras aplicarle el teorema del valor final se obtiene:
\begin{equation}
	e(\infty)=\lim_{s \rightarrow 0}
	s H_{e, D|PID}(s) R(s)
\end{equation}

El límite tiende a 0 cuando $R(s)=\frac{1}{s}$ y para $R(s)=\frac{1}{s^2}$. Para que este error sea $0$ cuando $R(s)=\frac{1}{s^3}$, se debe cumplir la siguiente condición:
\begin{equation}
K_P \tau_{D2} = \frac{p}{K}
\end{equation}

\begin{thebibliography}{9}
\bibitem{git} \href{https://github.com/avicarioe/telelabo}{Repositorio del proyecto alojado en GitHub}
\bibitem{design} Félix Monasterio-Huelin y Álvaro Gutiérrez Martín;
\href{http://www.robolabo.etsit.upm.es/asignaturas/seco/apuntes/design.pdf}{Diseño}

\end{thebibliography}






\end{document}
